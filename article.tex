\documentclass[anonymous,sigplan,review,11pt,nonacm,natbib=false]{acmart}
\settopmatter{printfolios=false,printccs=false,printacmref=false}
\usepackage[maxnames=1,minnames=1,maxbibnames=100,natbib=true,citestyle=authoryear,bibstyle=authoryear,doi=false,url=false,isbn=false,isbn=false]{biblatex}

\usepackage[russian]{babel}

\usepackage{rotating}
\usepackage{debate}
\usepackage{ffcode}
\usepackage{mathtools}
\usepackage{amsmath}
\usepackage{MnSymbol}
\usepackage[capitalize]{cleveref}
\usepackage{paralist}
\usepackage{cancel}
\usepackage{to-be-determined}
\usepackage{tikz}

\acmBooktitle{untitled}
\title{Strings functions analysis}
\author{Yegor Bugayenko}
\email{yegor256@gmail.com}
\affiliation{Huawei\city{Moscow}\country{Russia}}
\ccsdesc[100]{Object-Oriented Programming}
\keywords{Object-Oriented  Programming}

\begin{document}
    \raggedbottom

    \begin{abstract}
    \end{abstract}

    \maketitle

    \section{Introduction}\label{sec:intro}

    A string is an ordered set of characters. The standard library of string functions varies from language to language. Then in the case of creating a new language, you need to build your own library based on some criteria. However, the number of different functionality functions for strings that can be invented is at least countable (to put it simply - infinite). Then what to base on to identify the necessary set of functions for an object-oriented language? \hl{<- RQ}

    Most programming languages add functions to their string libraries without any limit on the number, for this reason there is no such research on this topic at the moment.

    This article aims to make a multivariate analysis of different parameters of string functions.

    We should keep in mind that the number of string functions in different programming languages is very large, and this number can grow, as we said before, without limitations. For this reason, this article does not consider all existing functions, but only those that meet some criteria. \hl{<- Transfer to method}

    \section{Related work}

    Among the other publications on this topic that exist at the time of this writing, all are of an applied nature. For this reason, they are not discussed in this article. In our article, we provide data that can be used in a free form, including for creating our own string libraries.

    \section{Considered functions}

    In order to investigate functions, we have to make a list of functions to be investigated. For this purpose, programming languages such as Java, Python, C++, Ruby were chosen. They were chosen on the basis that they are quite popular, and they implement different paradigms (e.g. strict and non-strict typing in C++ and Python respectively).

    A table with the list of different functions from standard string libraries for these languages has been compiled. For the analysis will be selected functions, which are presented at least in two programming languages (out of four presented). Since for the functions presented only in one language, we can conclude that they are used in too special or rare cases.

    To avoid self-repeats, a summary list of the functions to be analyzed is presented in Section 8.

    \section{Popularity of functions}

    \subsection{Definition}

    Since popularity is a measurable parameter, let's calculate the number of uses of each string function (see below). Then we denote popularity as a percentage of the number of uses of a certain string function relative to the total number of uses of all string functions. This approach makes it possible to estimate the frequency of string functions usage exactly in the context of correlation with other string functions.

    \subsection{Search algorithm}

    The GitHub and Sourcegraph platforms were chosen to calculate the number of uses of the functions. This platform provides the ability to search for any words or phrases in all GitHub projects, as well as if necessary to specify other parameters.

    However, you should take into account that asking for the number of times a certain word occurs in the code in certain cases will not give a correct result, there are a number of reasons for that.

    \subsubsection{The word is a comment} \hl{}

    For some words, it is more common for them to appear in files as names/comments/inside strings rather than as called functions. Since the article deals with the use of string functions for object-oriented languages, and they use dot notation, it will be more efficient to make queries like .<function name>, instead of <function name>.

    \subsubsection{It's a another class function} \hl{}

    In this kind of analysis, for some functions there is a problem that it is not clear which class of objects has which function. For example, the contains function is present in string as well as in most collections.

    Of course, it makes sense to examine the context in which the functions were used. However, this task is difficult because of the large amount of data to be analyzed.

    It is worth noting that functions which have had this problem are popular, this we have obtained by making queries about the number of times they occur for code bases. For this reason, this data will not be counted in this article.

    \section{Mutual usages}

    For any function (in our case string functions) you can consider such a parameter as its use by other functions. It is important because it is impossible to exclude a function from the library in case many other functions of the library use it.

    This parameter will be an integer numeric value that will be equal to the number of functions that can use it in their implementation (that is, it should not just be present in the implementation, but be necessary there) without affecting the asymptotics.

    \section{Asymptotics}

    \subsection{Mutable}

    In order to define the asymptotics of the different string functions, let us define what a string is. Since we are considering mutable strings, we consider a string to be an array of characters with which we can use write and read operations. This way we can avoid copying for every mutable operation.

    \subsection{Immutable}

    Immutable strings impose a number of constraints on the operation. We define immutable strings as an array of characters where we are allowed to read characters in a constant time, and each mutable operation creates a new copy. We are also allowed to store a constant number of other variables, such as various flags.

    \section{Finding the minimum set}

    \subsection{Necessity}

    Even if there are various criteria for functions, it is still necessary to determine how to get a library based on them. Therefore, we propose an algorithm that can be used for various (including not yet existing) object-oriented programming languages.

    \subsection{Parameter setting}

    It is necessary to set a criterion on the basis of which a decision will be made whether to add a function to the current set of functions.

    \subsection{Algorithm}

    Now specifically about the algorithm on the basis of which the final set of functions will be obtained.

    \begin{enumerate}
        \item Set is initially empty.
        \item Functions are considered in priority order, by default the most popular functions come first. If the function meets criterion 7.2, it is added to the set.
        \item The second step is repeated until there are no unselected functions left.
        \item In reverse order of priority the functions in the set are considered and those that don't meet criterion 7.2 are removed.
    \end{enumerate}

    \section{Results}

    Below is a table with the results - metrics.

    \begin{table*}[]
        \centering
        \begin{tabular}{|c||c|c|c|c|}
            \hline
            & Popularity & Mutual usages & Mutable O(?) & Immutable O(?) \\ \hline \hline
            capitalize &  &  & 1 & 1 \\ \hline

            char-at &  &  & 1 & 1 \\ \hline

            compare &  &  & n & n \\ \hline

            compare-ignore-case &  &  &  & \\ \hline

            concatenate &  &  & n & n \\ \hline

            containts &  &  & n & n \\ \hline

            count &  &  &  & \\ \hline

            ends-with &  &  & n & n \\ \hline

            equals &  &  & n & n \\ \hline

            find-first &  &  & n & n \\ \hline

            find-last &  &  & n & n \\ \hline

            format &  &  & n & n \\ \hline

            hash &  &  & n & n \\ \hline

            insert &  &  & n & n \\ \hline

            is-alphabetic &  &  & n & n \\ \hline

            is-empty &  &  & 1 & 1 \\ \hline

            is-hexadecimal &  &  & n & n \\ \hline

            is-int &  &  & n & n \\ \hline

            is-lowercase &  &  & n & n \\ \hline

            is-uppercase &  &  & n & n \\ \hline

            is-whitespace &  &  & n & n \\ \hline

            join &  &  & n & n \\ \hline

            length &  &  & 1 & 1 \\ \hline

            matches-regexp &  &  & n & n \\ \hline

            partition &  &  & n & n \\ \hline

            repeat &  &  & n & n \\ \hline

            replace-all &  &  & n & n \\ \hline

            replace-first  &  &  &  & \\ \hline

            split &  &  & n & n \\ \hline

            starts-with &  &  & n & n \\ \hline

            strip &  &  & n & n \\ \hline

            substring &  &  & n & 1 \\ \hline

            swap-case &  &  & n & n \\ \hline

            to-float &  &  & n & n \\ \hline

            to-int &  &  & n & n \\ \hline

            to-lowercase &  &  & n & n \\ \hline

            to-uppercase &  &  & n & n \\ \hline

            trim-left &  &  & n & n \\ \hline

            trim-right &  &  & n & n \\ \hline

            reverse &  &  & n & 1 \\ \hline

            remove &  &  &  & \\ \hline

            uncapitalize &  &  &  & \\ \hline

            as-bytes &  &  &  & \\ \hline

            chop &  &  &  & \\ \hline

            clear &  &  &  & \\ \hline

% \begin{sideways} \end{sideways}

        \end{tabular}
        \caption{String functions criteria}
        \label{tab:my_label}
    \end{table*}

    Based on these data, we have compiled our optimal library for string functions; the included functions are listed below.

    \hl{todo}

    \section{Discussion and analysis}

    \subsection{Mutual usages presenting}

    When we counted the Mutual usages parameter we used a numerical value. However, it is worth considering that a representation in the form of an oriented dependency graph would give more applied information. Because based on this graph, it would be less biased to construct a minimal set of string functions.

    However, it is not possible to represent the graph in this paper because the number of its edges would be too large (of the order of the square of the number of functions). And such representation would lose any comprehensible visualization.

    \subsection{Problems with function intersection}

    Earlier we mentioned the problem with popularity calculation for functions that have the same name, but are called on objects of different classes.

    To solve this problem we can apply the extrapolation function here.

    Note that this function does not show absolute accuracy; however, if the files to be analyzed are selected at random, the error decreases very much.

    In order to estimate the error it is necessary to take several similar samples and compare how different the results are.

    \subsection{Unexpected advantage of immutability}

    An interesting fact was that, judging by the popularity of the use, the immutable methods are superior to the mutable ones.

    \section{Conclusion}\label{sec:conclusion}

    \hl{todo}

    \section{Used materials}

    https://yegor256.github.io/plum/

    \hl{todo}

    \printbibliography

\end{document}